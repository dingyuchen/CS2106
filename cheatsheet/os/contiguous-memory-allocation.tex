\section{Contiguous Memory Allocation}
\subsection*{Contiguous Memory Management}
\emph{Fixed-size partition:} (-) internal fragmentation. (-) external fragmentation (total unused memory > request size but no partition fits)\\
\emph{Dynamic partition:} OS keeps track of occupied and free regions, performs splitting and merging. (+) no internal fragmentation (by splitting). (-) external fragmentation\\
\emph{Relocation \& compaction:} Move partitions around to remove external fragmentation. (-) data which are addresses may refer to wrong address after relocation.

\subsection*{Dynamic Memory Allocation Algorithms}
\emph{Sequential Fit:} In free block: doubly linked list+size=~3 words for bookkeeping (minimum free block size)\\
First fit: (-) front list split more, many small blocks at start\\
Best fit: find smallest fitting block. (-) need searching whole list. (-) bad external fragmentation with many tiny blocks. (+) experimentally good memory use results.\\
Next fit: search from previously searched position. (+) avoids accumulation of small blocks at start. (-) poor locality affects caches.\\
Freeing a block: boundary tag optimization can coalesce in constant time.\\
\emph{Buddy System:}\\
avail[i] records a free list of blocks of size $2^i$ (NULL means no free blocks). Splitting $xx..xx00..00$ ->$xx..xx00..00$ and $xx..xx10..00$. Find buddies for block of size $2^i$ -> flip bit at position $i+1$ from right to find buddy address. May have smallest allocated block size. (avail[0] for illustration)\\
Buddy allocation n: find smallest i for $2^i \geqslant n$. If avail[i] not empty, select free block there; else, find free block $b$ from smallest avail[k] where $k>i$, split $b$ until avail[i] has block.\\
Buddy deallocation p: find buddy p'. If p' not free, add p to avail list; else, coalesce p and p' and free it recursively.